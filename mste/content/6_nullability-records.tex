% Nullable Reference Types: zwar prüfungsrelevant, aber nur einfache Codierungsaufgaben

\section{Nullability \& Records}

\begin{multicols*}{2}
\subsection{Nullability Value Types}
\subsubsection{Nullable struct}
\begin{lstlisting}
public struct Nullable<T> 
    where T : struct
{   
    public Nullable(T value);

    public bool HasValue { get; }
    public T Value { get; }
}
\end{lstlisting}
\begin{tabular}{| c | c | c |}
\hline
Status & HasValue & Value\\ \hline\hline
null & false & ? \\ \hline
not null & true & T\\
\hline
\end{tabular}
\vspace*{2mm}
\begin{itemize}
    \item HasValue ist true: Liefert den Wert
    \item HasValue ist false: System.InvalidOperationException
\end{itemize}
\subsubsection*{T? Syntax}
Reines Compiler-Feature, Verbessert die Lesbarkeit, Direkte Zuweisung von null möglich.
\begin{lstlisting}
int? x = 123; 
int? y = null;

// Compiler-Output
System.Nullable<int> x = new System.Nullable<int>(123);
System.Nullable<int> y = new System.Nullable<int>();
\end{lstlisting}
\fat{Sicheres Lesen:}
\begin{lstlisting}
// Klassisch
int x1 = x.HasValue 
    ? x.Value
    : default;

// Via Methode
int x2 = x.GetValueOrDefault();

// Via Methode inkl. eigenem Default
int x3 = x.GetValueOrDefault(-1);
\end{lstlisting}

\subsection{Nullability Reference Types}
Variable kann mit ? annotiert werden: 
\begin{lstlisting}
string s1; // Non-nullable
string? s2; // Nullable
\end{lstlisting}
Null-forgiving Operator ! übersteuert den null-state explizit:
\begin{lstlisting}
string s3 = null!;
\end{lstlisting}
Extrem gefährlich, nur mit Bedacht einsetzen. Compile Time ≠ Runtime (inkonsistent)

\subsection{Operatoren}
\subsubsection{is null / is not null}
\begin{lstlisting}
int? x = null; 
string s = null;

bool b1a = x is null;
bool b1b = x is not null;
bool b2a = s is null;
bool b2b = s is not null;

// Compiler-Output
bool b1a = x.HasValue; bool b1b = !x.HasValue;
bool b2a = object.ReferenceEquals(x, null);
bool b2b = !object.ReferenceEquals(x, null);
\end{lstlisting}
\subsubsection{null-coalescing}
Gibt den linken Teil zurück, sofern nicht null, Anstelle von -1 könnte auch eine throw Anweisung stehen.
\begin{lstlisting}
int? x = null;
int i = x ?? -1;

// Compiler-Output (vereinfacht)
int i = x is not null
    ? x.GetValueOrDefault()
    : -1;  
\end{lstlisting}
\subsubsection{null-coalescing assignment}
Gibt den linken Teil zurück, sofern nicht null.
\begin{lstlisting}
int? x = null;
i ??= -1;

// Compiler-Output (vereinfacht)
if (i is null)
{
    i = -1; 
}
\end{lstlisting}
\subsubsection{null-conditional}
Führt den rechten Teil aus, sofern der linke Teil nicht null ist. Andernfalls wird ein Default-Wert geliefert.
\begin{lstlisting}
object o = null; 
Action a = null;

string s = o?.ToString();
a?.Invoke();

// Compiler-Output
string s = o is not null
    ? o.ToString()
    : default;
if (a is not null)
{
    a(); 
}
\end{lstlisting}

\subsection{Record Types}
Reine Datenrepräsentations-Klassen. Record Types sollten immutable sein. Automatisch generierte Members:
\begin{itemize}
    \item Konstruktor
    \item Properties (init only setter)
    \item Value equality
    \item Darstellung (ToString-Methode, etc.)
    \item Vererbung wird berücksichtigt (z.B. bei Equality)
    \item Bei Vererbung: Base muss Record sein
\end{itemize}
\begin{lstlisting}
public record [class|struct] Person( int Id, string Name);

public abstract record Person(int Id); public record SpecialPerson(
    int Id,
    string Name) : Person(Id);
\end{lstlisting}
Vereinfacht das erzeugen bon Kopien
\begin{lstlisting}
Person p1 = new(0, "Mary");
Person p2 = p1 with { Id = 1 };
bool eq2 = p1 == p2; // false
Person p3 = p1 with { };
bool eq3 = p1 == p3; // true
\end{lstlisting}
\end{multicols*}