% Ohne Operatoren

\section{Klassen \& Structs}
\begin{multicols*}{2}
\subsection{Klassen}
\begin{itemize}
    \item ReferenceType
    \item Ableiten von Basisklasse möglich
    \item Verwendung als Basisklasse möglich
    \item Implementieren von Interfaces möglich
    \item Konstruktoren: Parameter optional
    \item Felder: Initialisierung erlaubt
\end{itemize}
\subsubsection{Statisch}
\fat{Regeln:}
\begin{itemize}
    \item Nur statische Members erlaubt
    \item Kann nicht instanziert werden
    \item Sind sealed
\end{itemize}
\fat{Zweck:}
\begin{itemize}
    \item Sammlung an Standard-Werte. Beispiel: MyMath.Pi
    \item Sammlung an Funktionalität Beispiel: MyMath.Sin(...)
    \item Definition von Erweiterungsmethoden
\end{itemize}
\subsubsection{Nested Types}
Für Hilfsklassen innerhalb einer anderen Klasse gedacht.
\nfat{Regeln:}
\begin{itemize}
    \item Äussere hat Zugriff auf public members der Inneren
    \item Innere hat Zugriff auf alles der Äusseren
    \item Fremde Klassen erhalten Zugriff auf innere Klasse, wenn diese public ist
    \item Erlaubte Nested Types sind: Klassen, Interfaces, Structs, Enums, Delegates
\end{itemize}

\subsection{Structs}
\begin{itemize}
    \item ValueType: Auf Stack oder in-line in Objekt auf Heap
    \item Ableiten von Basisklasse nicht möglich
    \item Verwendung als Basisklasse nicht möglich
    \item Implementieren von Interfaces möglich
    \item Konstruktoren: mind. 1 Parameter
    \item Felder: Initialisierung nicht erlaubt
\end{itemize}
\fat{Verwendung nur wenn folgende Umstände:} 
\begin{itemize}
    \item Repräsentiert einen einzelnen Wert
    \item Instanzgrösse ist kleiner als 16 Byte (128 Bit)
    \item Ist \dq immutable\dq
    \item Wird nicht häufig geboxt
    \item Entweder kurzlebig oder wird in andere Objekte eingebettet
\end{itemize}

\subsection{Instanzierung}
Erzeugt aus einer Klasse ein Objekt.
\begin{lstlisting}
Stack s = new Stack(10); //Klassisch
Stack s = new(10); //Target-typed new expressions (C\#10)
\end{lstlisting}

\subsection{Felder}
\fat{Feld:}
\begin{itemize}
    \item Initialisierung in Deklaration optional
    \item In Deklarationsreihenfolge initialisiert
    \item Initialisierung darf nicht auf Felder und Methoden zugreifen
\end{itemize}
\fat{Konstante:}
\begin{itemize}
    \item Muss Initialisierungswert haben
    \item Wert muss zur Compilezeit berechenbar sein
\end{itemize}
\fat{Readonly:}
\begin{itemize}
    \item In Deklaration oder Konstruktor initialisiert
    \item Muss nicht zur Compilezeit berechenbar sein
    \item Wert darf später nicht mehr geändert werden
\end{itemize}
\begin{lstlisting}
int _value = 0;
const long Size = int.MaxValue / 3 + 1234;
readonly DateTime _date1 = DateTime.Now; 
readonly DateTime _date2;
\end{lstlisting}


\subsection{Methoden}
\subsubsection{Parameter}
\fat{Value-Parameter(call by value):} Kopie des Stack-Inhaltes wird übergeben
\nfat{ref-Parameter(call by reference):} 
\begin{itemize}
    \item Adresse der Variable wird übergeben
    \item Variable muss initialisiert sein
    \item Es muss eine Variable übergeben werden
\end{itemize}
\fat{out-Parameter:}
\begin{lstlisting}
void Init(out int a, out int b) { a = 1; b = 2; }

// Bisher
int a, b;
Init(out a, out b); // a == 1 / b == 2

// C# 7.0 regulär
Init(out int a1, out int b1);
Console.WriteLine($"{a1} / {b1}");

// C# 7.0 Parameter ignoriert
Init(out int a2, out _); Console.WriteLine($"{a1}");

// Praxisbeispiel
bool success1 = int.TryParse("123a", out int i1);
bool success2 = int.TryParse("123a", out _);
\end{lstlisting} 
\fat{params-Array:}
\begin{itemize}
    \item Erlaubt beliebig viele Parameter (0-n)
    \item Muss am Schluss der Deklaration stehen
    \item Nur ein params-Array erlaubt
    \item params darf nicht mit ref / out kombiniert werden
\end{itemize}
\begin{lstlisting}
void Sum(out int sum, params int[] values) {
    sum = 0;
    foreach (int i in values) sum += i;
}
\end{lstlisting}
\fat{Optional:}
\begin{lstlisting}
private void Sort( 
int[] array, //Erforderlich
int from = 0, //Optional
int to = -1, //Optional
bool ascending = true,  //Optional
bool ignoreCase = false //Optional
){}
\end{lstlisting}
\fat{Named:}
\begin{lstlisting}
int[] a = { 3, 5, 2, 6, 8, 4 };
Sort(a, ascending: false);
Sort(a, ignoreCase: true, from: 3);
Sort(a, ignoreCase: true, ascending: false,
to: 2, from: 3);
Sort(a, ignoreCase: true, ascending: false,
from: 2, to: 3); Sort(ignoreCase: true, ascending: false,
from: 2, to: 3, array: a);
\end{lstlisting}
\subsubsection{Überladung}
Mehrere Methoden mit dem gleichen Namen.
\begin{itemize}
    \item Unterschiedliche Anzahl Parameter
    \item Unterschiedliche Parametertypen oder
    \item Unterschiedliche Parameterarten (ref / out)
    \item Der Rückgabetyp spielt bei der Unterscheidung keine Rolle
    \item Default Parameter kein Unterscheidungsmerkmal
    \item Methoden ohne optionale Parameter haben Vorrang
\end{itemize}

\subsection{Properties}
\fat{Regeln:}
\begin{itemize}
    \item Read- und/oder Write-Only möglich (get oder set weglassen)
    \item Schlüsselwort value für Zugriff auf Wert in Setter
    \item Sichtbarkeiten von Getter / Setter können verändert werden
\end{itemize}
\begin{lstlisting}
// Backing-Field
private int _length;

// Property
public int Length 
{
    get { return _length; }
    set { _length = value; }
}

// Read-only Property
public int LengthReadOnly
{
    get { return _length; }
}

// Write-only Property
public int LengthWriteOnly
{
    set { _length = value; }
}

// Private-Set Property
public int LengthPrivateSet
{
    get { return _length; }
    private set { _length = value; }
}

// Auto Property, Getter/Setter beide zwingend
public int LengthAuto { get; set; }

// Auto Property initialized, set darf weggelassen werden
public int LengthInit { get; } = 5;
\end{lstlisting}
\subsubsection{Objekt-Initialisierung}
\label{sec:object_initialisation}
Properties können direkt initialisiert werden:
\begin{lstlisting}
MyClass mc = new()
{
    Length = 1,
    Width = 2 
};
\end{lstlisting}
\fat{Init-Only setter:} Property muss während Initialisierung gesetzt werden:
\begin{itemize}
    \item Im Konstruktor des umschliessenden Typs
    \item Object Initialisierung
    \item with-Initializer (siehe Record Types)
    \item Indirekt über anderes init-only Property
\end{itemize}
\begin{lstlisting}
// Init-only setter
public int LenthInitOnly { get; init; }
// Anwendungsbeispiel
MyClass mc = new()
{
    LenthInitOnly = 1 
};
\end{lstlisting}

\subsubsection{Expression-Bodied Members}
\begin{lstlisting}
public class Examples { 
    private int _value;
    
    // Constructors / Destructors (C# 7.0)
    public Examples(int v) => _value = v;
    ~Examples() => _value = 0;
    
    // Methods (C# 6.0)
    public int Sum(int x, int y) => x + y; 
    public int GetZero() => 0;
    public void Print() => Console.WriteLine("...");
    
    // Properties (C# 6.0)
    public int Zero => 0;
    public int Bla => Sum(Zero, 2);
    
    // Getters/Setters (C# 7.0)
    public int Value 
    { 
        get => _value;
        set => _value = value;
    }
}
\end{lstlisting}

\subsubsection{Indexer}
Überladen des Index-Operators.
\begin{itemize}
    \item Read- und/oder Write-Only möglich
    \item Schlüsselwort this definiert den Indexer
\end{itemize}
\begin{lstlisting}
class MyClass {
    private string[] _arr = new string[10];
    // int-Indexer
    public string this[int index]
    {
        get { return _arr[index]; }
        set { _arr[index] = value; }
    }
    // string-Indexer
    public int this[string search]
    {   
        get {
            for(int i = 0; i < _arr.Length; i++) 
            { 
                if(_arr[i] == search) return i; 
            } 
            return -1;
        } 
    }
    // 2-Dimensionaler Indexer
    public string this[int i1, int i2]
    { 
        /* ... */ 
    } 
}
\end{lstlisting}

\subsection{Konstruktoren}
\begin{itemize}
    \item Aufruf anderer Konstruktoren möglich: this
    \item Aufruf auf Basis-Klassen-Konstruktor: base
\end{itemize}
\begin{lstlisting}
class MyClass {
    private int _x, _y;
    
    public MyClass() : this(0, 0) { } //Default
    public MyClass(int x) : this(x, 0) { }
    public MyClass(int x, int y)
    {
        _y = y; 
        _x = x;
    } 
}
\end{lstlisting}
\subsubsection{Default}
Keine Parameter
\begin{multicols*}{2}
    \fat{Klasse}
    \begin{itemize}
        \item Automatisch generiert wenn nicht vorhanden
        \item Wird nicht generiert, wenn anderer Konstruktor vorhanden
        \item Kann manuell implementiert werden
        \item Initialisiert Felder mit default(T)
        \item Konstruktor kann beliebig viele Felder initialisieren
    \end{itemize}
    \columnbreak
    \fat{Struct}
    \begin{itemize}
        \item Immer automatisch generiert
        \item Wird auch generiert, wenn anderer Konstruktor vorhanden
        \item Kann nicht manuell definiert werden
        \item Initialisiert Felder mit default(T)
        \item Konstruktur muss alle Felder initialisieren
    \end{itemize}
\end{multicols*}

\subsubsection{Statisch}
Identisch bei Klasse / Struct
\begin{itemize}
    \item Zwingend parameterlos
    \item Sichtbarkeit darf nicht angegeben werden
    \item Nur ein statischer Konstruktor erlaubt
    \item Wird genau einmal ausgeführt: Bei erster Instanzierung des Typen, Bei erstem Zugriff auf Statisches Feld des Typen
    \item Kann nicht explizit aufgerufen werden
\end{itemize}
\begin{lstlisting}
class MyClass {
    static MyClass()
    {
        /* ... */
    }
}
\end{lstlisting}
\subsubsection{Private}
Können nur intern verwendet werden. Es wird kein Default-Konstruktor erzeugt, wenn ein privater Konstruktor existiert
\subsubsection{Destruktor}
Ermöglichen Abschlussarbeiten beim Abbau eines Objekts.
\begin{itemize}
    \item Nur bei Klassen erlaubt, bei Structs verboten
    \item Zwingend parameterlos / ohne Sichtbarkeit
    \item Nur ein Destruktor erlaubt
    \item Wird nur vom Garbage Collector aufgerufen
    \item Destruktor der Basisklasse wird anschliessend ausgeführt
\end{itemize}
\begin{lstlisting}
~MyClass() { }
\end{lstlisting}

\subsubsection{Initialisierungs-Reihenfolge}
\fat{Ohne Vererbung:}
\begin{lstlisting}
class Base {
    private static int _baseStaticValue = 0;
    private int _baseValue = 0;
    static Base() { }
    public Base() { }
}

Base b1 = new Base();
    // Base > baseStaticValue
    // Base > Statischer Konstruktor 
    // Base > baseValue
    // Base > Konstruktor

Base b2 = new Base();
    // Base > baseValue 
    // Base > Konstruktor
\end{lstlisting}
\fat{Mit Vererbung:}
\begin{lstlisting}
class Base {
    private static int _baseStaticValue = 0;
    private int _baseValue = 0;
    static Base() { }
    public Base() { }
}

class Sub : Base
{
private static int _subStaticValue = 0; private int _subValue = 0;
static Sub() { }
public Sub() { }
}

Sub s1 = new Sub();
    // Sub > subStaticValue
    // Sub > Statischer Konstruktor 
    // Sub > subValue
    // Base > baseStaticValue
    // Base > Statischer Konstruktor 
    // Base > baseValue
    // Base > Konstruktor
    // Sub > Konstruktor

Sub s2 = new Sub();
    // Sub > subValue
    // Base > baseValue 
    // Base > Konstruktor 
    // Sub > Konstruktor
\end{lstlisting}
\includegraphics*[width=\columnwidth]{konstruktor}
\includegraphics*[width=\columnwidth]{konstruktor2}

\subsection{Operator Overloading}
\begin{itemize}
    \item Methode muss static sein
    \item Rückgabetyp ist frei wählbar
    \item Mindestens 1 Parameter muss vom Typ der enthaltenden Klasse sein
    \item Muss teilweise als Paar überladen werden:
    (==, !=), (<, >), (<=, >=), (true, false)
\end{itemize}
\begin{lstlisting}
class Point {
    private int _x, _y;
    public Point(int _x, int _y){ _x = _x;_y = _y; }

    public static Point operator ~(Point a) 
    {
        return new(a._x * -1, a._y * -1);
    }

    public static Point operator + (Point a, Point b)
    {
        return new(a._x + b._x, a._y + b._y);
    } 
}

// OK
public static int operator ~(Point a)
// Compilerfehler
public static Point operator ~(int a)
// OK
public static Point operator +(int a, Point b) 
public static Point operator +(Point a, int b) 
public static int operator +(Point a, Point b) 
// Compilerfehler
public static Point operator +(int a, int b)
\end{lstlisting}

\subsection{Partielle Klassen und Methoden}
Funktioniert mit Klassen, Structs, Interfaces
\begin{itemize}
    \item partial muss bei allen Teilen angemerkt sein
    \item Alle Teile müssen die gleichen Sichtbarkeitsattribute haben
\end{itemize}
\begin{lstlisting}
// File1.cs
partial class MyClass
{
    public void Test1() { }
}
// File2.cs
partial class MyClass
{
    public void Test2() { }
}
\end{lstlisting}
\end{multicols*}
