\section{Iteratoren}

\begin{multicols*}{2}
\subsection{foreach-Loop}
\begin{lstlisting}
//Syntax
foreach (ElementType elem in collection) { statement }

//Beispiel
int[] list = { 1, 2, 3, 4, 5, 6 };

foreach (int i in list)
{
    if(i == 3) continue; //Unterbricht aktuelle Iteration
    if(i == 5) break; //Unterbricht gesamten Loop
    Console.WriteLine(i);
}
\end{lstlisting}
\subsubsection{Kriterien für collection}
\begin{itemize}
    \item Muss IEnumerable / IEnumerable<T> implementieren
    \item Muss einer Implementation von IEnumerable / IEnumerable<T> \dq ähneln\dq:
    \begin{itemize}
        \item collection hat Methode GetEnumerator() mit Rückgabewert e
        \item e hat eine Methode MoveNext() mit Rückgabewert bool
        \item e hat ein Property Current
    \end{itemize}
\end{itemize}

\subsection{Iteratoren}
\subsubsection{Interfaces}
\begin{lstlisting}
//Non-generisch
public interface IEnumerable
{
    IEnumerator GetEnumerator();
}

public interface IEnumerator
{
    object Current { get; }
    bool MoveNext();
    void Reset();
}

//Generisch
public interface IEnumerable<out T>
    : IEnumerable
{
    IEnumerator<T> GetEnumerator();
}

public interface IEnumerator<out T> 
    : IDisposable, IEnumerator
{
    T Current { get; }
    //weitere Members werden vererbt
}
\end{lstlisting}
\subsubsection{Zugriff}
\begin{itemize}
    \item Mehrere aktive Iteratoren zur gleichen Zeit erlaubt
    \item Enumerator-Objekt muss Zustand vollständig kapseln, ansonsten Seiteneffekte
    \item Collection darf dementsprechend während Iteration nicht verändert werden
\end{itemize}
\subsubsection{Implementation}
Relativ aufwändig zu implementieren.
\fat{Non-Generisch:}
\begin{lstlisting}
class MyList : IEnumerable {
    public IEnumerator GetEnumerator()
    { return new MyEnumerator(); }

    class MyEnumerator : IEnumerator
    {
        public object Current
        { get { /* ... */ } }
        public bool MoveNext() { /* ... */ }
        public void Reset() { /* ... */ }
    }
}   

//Anwendung
MyList c = new();
foreach (object elem in c)
{
    /* ... */
}
\end{lstlisting}
\fat{Generisch:}
\begin{lstlisting}
class MyIntList : IEnumerable<int> {
    public IEnumerator<int> GetEnumerator()
    { return new MyIntEnumerator(); }
    
    //Muss auch implementiert werden, da IEnumerable<int> von IEnumerable ableitet
    IEnumerator IEnumerable.GetEnumerator() 
    { return GetEnumerator(); }
    
    class MyIntEnumerator : IEnumerator<int>
    {   
        public int Current
        { get { /* ... */ } }
        object IEnumerator.Current
        { get { /* ... */ } }

        public bool MoveNext() { /* ... */ }
        public void Reset() { /* ... */ }
        public void Dispose() { /* ... */ }
    } 
}

//Anwendung
MyIntList c = new();
foreach (int elem in c)
{
    /* ... */
}
\end{lstlisting}
\subsubsection{Vereinfachung durch Iterator-Methoden}
\fat{Signatur:}
\begin{itemize}
    \item public IEnumerator GetEnumerator()
    \item public IEnumerator<T> GetEnumerator()
    \item Beinhaltet mindestens ein yield return Statement
\end{itemize}
\begin{lstlisting}
class MyIntList 
{
    public IEnumerator<int> GetEnumerator()
    {
        yield return 1;
        yield return 3;
        yield break;
        yield return 6;
    }
}

//Anwendung
MyIntList c = new();
foreach (int elem in c)
{
    /* ... */
}

\end{lstlisting}
\fat{yield return:\\}
Gibt den nächsten Wert für die nächste Iteration eines foreach Loops zurück.
\begin{lstlisting}
yield return expr;  
yield break; //Terminiert Iteration  
\end{lstlisting}
Typ von expr:
\begin{itemize}
    \item T: wenn IEnumerator<T>
    \item object: wenn IEnumerator
\end{itemize}
\subsection{Spezifische Iteratoren}
\begin{lstlisting}
class MyIntList 
{
    private int[] _x = new int[10];
    
    //Standard Iterator
    public IEnumerator<int> GetEnumerator() 
    {
        for (int i = 0; i < _x.Length; i++)
            yield return _x[i];
    }
    
    //Spezifische Iterator-Methode
    //Rückgabewert ist IEnumerable<T> nicht IEnumerator<T>
    public IEnumerable<int> Range(int from, int to) 
    { 
        for (int i = from; i < to; i++)
            yield return _x[i];
    }
    
    //Spezifisches Iterator-Property
    //Rückgabewert ist IEnumerable<T> nicht IEnumerator<T>
    public IEnumerable<int> Reverse 
    {
        get {
            for (int i = _x.Length - 1; i >= 0; i--)
                yield return _x[i];
        } 
    }
}

MyIntList list = new();

//Anwendung Range
foreach (int elem in list.Range(2, 7))
{
    /* ... */
}

//Anwendung Reverse
foreach (int elem in list.Reverse)
{
    /* ... */
}
\end{lstlisting}

\subsection{Extension Methods}
\begin{itemize}
    \item Erlaubt das Erweitern bestehender Klassen (aus Anwendersicht)
    \item Signatur der Klasse wird nicht verändert
    \item Aufruf sieht jedoch so aus, als wäre es eine Methode auf der Klasse
\end{itemize}
\fat{Regeln:}
\begin{itemize}
    \item Kein Zugriff auf interne Members aus Extension Methode
    \item ToStringSafe ist nur sichtbar, wenn der Namespace von ExtensionMethods importiert wurde
    \item Bei Konflikt mit Methode auf der Zielklasse gewinnt immer die der Zielklasse
    \item Erlaubt auf Klassen / Structs / Interfaces / Delegates / Enumeratoren / Arrays
\end{itemize}
\subsubsection{Deklaration}
\begin{itemize}
    \item Muss in statischer Klasse deklariert sein
    \item Muss static sein
    \item Erster Parameter: this Schlüsselwort und Angabe auf welcher Klasse die Methode verwendbar ist
\end{itemize}
\begin{lstlisting}
using Extensions;

public static class ExtensionMethods 
{
    static string ToStringSafe(this object obj) 
    {
        return obj == null
            ? string.Empty : obj.ToString();
    }
}
\end{lstlisting}

\subsection{Deferred Evaluation (Lazy-Loading)}
Aufruf von GetEnumerator() oder von Range(...) führt Iterator-Methode noch NICHT aus, 
erst der Aufruf von IEnumerator<T>.MoveNext().\\
Dies passiert implizit im foreach Loop
\begin{lstlisting}
IEnumerator enumerator = list .GetEnumerator();

try
{
    while (enumerator.MoveNext()) {
        int i = (int)enumerator.Current;
        /* ... */
    } 
}
finally { /* ... */ }   
\end{lstlisting}
\subsubsection{Extension Methods und Iteratoren}
Erweitern der IEnumerable Interfaces um Query Operatoren über Extension Methods.
\begin{itemize}
    \item Jede Collection welche IEnumerable implementiert kann auf diese Operatoren zugreifen
    \item Liefern in der Regel wieder ein IEnumerable
    \item Können mit dem \dq.\dq-Operator verkettet werden
    \item Werden deferred evaluiert
\end{itemize}
\begin{lstlisting}
static class OstExtensions {
    public static IEnumerable<T> OstWhere<T>(
        this IEnumerable<T> source,
        Predicate<T> predicate)
    {
        foreach (T item in source) {
            if (predicate(item))
                yield return item;
        } 
    }

    public static IEnumerable<T> OstOfType<T>(
        this IEnumerable source)
    {
        foreach (object item in source)
        {
            if (item is T)
                yield return (T)item;
        } 
    }
}
//Verwendung
object[] list = { 4, 3.5, "abc", 3, 6 };
IEnumerable<int> res = list
    .OstOfType<int>()
    .OstWhere(k => k % 2 == 0);

foreach (int i in res) { Console.WriteLine(i); }
\end{lstlisting}


\end{multicols*}