\section{Delegates \& Events}

\begin{multicols*}{2}

\subsection{Delegates}
\begin{itemize}
    \item Bietet eine Vereinfachung von Interfaces. 
    \item Typsichere Funktions-Pointer
    \item Reference Type: Abgeleitet von System.MulticastDelegate
    \item Hält intern die Referenz auf 0-n Methoden
    \item \fat{Verwendung:} Methoden als Parameter übergeben oder Definition von Callback-Methoden
\end{itemize}
\subsubsection{Deklaration \& Zuweisung}
\begin{itemize}
    \item Schlüsselwort delegate
    \item Definition Rückgabewert (hier void)
    \item Definition eines Namens (hier Notifier)
    \item Definition von Parametern
\end{itemize}
\begin{lstlisting}
// Deklaration eines Delegate-Typs
public delegate void Notifier(string sender);

class Examples
{
    public static void Test() {
        // Deklaration Delegate-Variable
        Notifier greetings;
        // Zuweisung einer statischen Methode
        greetings = new Notifier(MyClass.SayHi);
        // Zuweisung einer Instanz-Methode
        MyClass myClass = new();
        greetings = new Notifier(myClass.SayInstance);
        // Kurzform (statisch oder Instanz)
        greetings = MyClass.SayHi;
        greetings = myClass.SayInstance;
        // Aufruf einer Delegate-Variable
        greetings("John");
    }
}

public class MyClass
{
    public static void SayHi(string sender)
    {
        Console.WriteLine("Hello {0}", sender); 
    }

    public void SayInstance(string sender) {
        Console.WriteLine("Inst. {0}", sender);
    }
}

//Jede Methode mit passender Signatur erlaubt 
//Identische Anzahl Parameter / Parametertypen / Rückgabetyp / Parameterarten (ref,out,value)
greetings = SayHi;
greetings = Console.WriteLine;
greetings = null; //Zuweisung von null erlaubt
greetings("John"); //Exception

//Aufruf Syntax
object result = delegateVar[.Invoke](params);

//delegateVar darf nicht null sein
if (delegateVar != null) 
{
    delegateVar(params);
}
//Ab C# 6.0 
delegateVar?.Invoke(params);
\end{lstlisting}
\includegraphics*[width=\columnwidth]{delegates}
\includegraphics*[width=\columnwidth]{delegates2}
\subsection{Multicast-Delegates}
\begin{itemize}
    \item Jeder Delegate-Typ ist ein Multicast Delegate
    \item Delegate-Variable kann beliebig viele Methoden-Referenzen enthalten
    \item Zuweisung mit $=$  $\rightarrow$ Setzt Inhalt der Delegate-Variable neu
    \item Zuweisung mit $+=$  $\rightarrow$ Fügt der Delegate-Variable Methoden hinzu
    \item Zuweisung mit $-=$  $\rightarrow$ Entfernt Methoden aus Delegate-Variable
\end{itemize}
\begin{lstlisting}
public delegate void Notifier(string sender);

class Examples
{
    public static void Test()
    {
        Notifier greetings; 
        greetings = SayHi; 
        greetings += SayCiao;
        greetings("John");
    }
    
    private static void SayHi(string sender)
    {
        Console.WriteLine("Hello {0}", sender);
    }
    
    private static void SayCiao(string sender)
    {
        Console.WriteLine("Good bye {0}", sender); 
    }
}
\end{lstlisting}
Interner Aufbau als Linked-List:
\begin{itemize}
    \item target: Referenz auf Zielobjekt
    \item method: Methoden-Referenz
    \item prev: Referenz auf letztes Delegate
\end{itemize}
\includegraphics*[width=.75\columnwidth]{mdelegates}
\fat{Compiler-Output für das Verständnis:}
\begin{lstlisting}
public delegate void Notifier(string sender); 
//wird zu
public sealed class Notifier : MulticastDelegate
{
    public Notifier(object obj, int method) { }
    public virtual void Invoke(string sender) { }
    public virtual IAsyncResult BeginInvoke(
        string sender,
        AsyncCallback callback,
        object obj) {}
    public virtual void EndInvoke(IAsyncResult result) { } 
}
//----------------------------------------------
greetings("John");
//wird zu
foreach(Delegate del in greetings.GetInvocationList())
{
    ((Notifier)del).Invoke("John");
}
\end{lstlisting}
\subsubsection{Verwendungsbeispiel Action}
Methode ForAll soll ein int Array entgegen nehmen und auf jedem Element eine Funktion ausführen.
\begin{lstlisting}
public delegate void Action(int i);

public class MyClass
{
    public static void PrintValues(int i)
    { Console.WriteLine("Value {0}", i); }

    public void SumValues(int i) { Sum += i; }
    
    public int Sum { get; private set; }
}

public class FunctionParameterTest
{
    static void ForAll(int[] array, Action action)
    {
        Console.WriteLine("ForAll called...");
        if (action == null) { return; }
        foreach (int t in array)
        {
            action(t); 
        }
    } 
}

public static void TestPrintValues()
{
    MyClass c = new();
    int[] array = { 1, 5, 8, 14, 22 };
    
    // Delegate Variables
    Action v1 = MyClass.PrintValues; // Static
    Action v2 = c.SumValues; // Instance Method

    // Execution
    ForAll(array, v1);
    ForAll(array, v2);
    Console.WriteLine("--- Sum {0}", c.Sum);
}
\end{lstlisting}

\subsection{Generische Delegates}
\begin{lstlisting}
public delegate void Action<T>(T i);

public class MyClass
{
    public static void PrintValues<T>(T i)
    { Console.WriteLine("Value {0}", i); }
}

public class FunctionParameterTest
{
    static void ForAll<T>(T[] array, Action<T> action)
    {
        Console.WriteLine("ForAll called...");
        if (action == null) { return; }
        foreach (T t in array)
        {
            action(t); 
        }
    } 
}

public static void TestPrintValues()
{
    MyClass c = new();
    int[] a1 = { 1, 5, 8 };
    string[] a2 = {"a", "b", "c"};
    ForAll(a1, MyClass.PrintValues);
    // Funktioniert
    ForAll(a2, MyClass.PrintValues);
}
\end{lstlisting}
\fat{Framework Class Library}
\begin{lstlisting}
//Aktion mit 0-n Parametern, void als Rückgabewert
public delegate void Action();
public delegate void Action<in T>(T obj);
public delegate void Action<in T1, in T2>(T1 obj1, T2 obj2);

//Funktion mit 0-n Parametern, TResult als Rückgabewert
public delegate TResult Func<out TResult>();
public delegate TResult Func<in T, out TResult>(T arg);
public delegate TResult Func<in T1, in T2, out TResult>(T1 arg1, T2 arg2);

//Prädikat mit 1 Parameter, bool als Rückgabewert
public delegate bool Predicate<in T>(T obj);

//Generisches Delegate für Event Handling im .NET Framework. Definiert Sender sowie EventArgs
public delegate void EventHandler<TEventArgs>(object sender, TEventArgs e);
public delegate void EventHandler(object sender, EventArgs e);
\end{lstlisting}
\subsection{Callback}
\fat{Besser mit Events lösen!}
\\Tickende Uhr
\begin{itemize}
    \item Dynamisch wählbarer Intervall
    \item Benachrichtigung beim Ticken
    \item Subscribe / Unsubscribe Logik
\end{itemize}
\fat{Clock:}
\begin{lstlisting}
public delegate void TickEventHandler (int ticks, int interval);

public class Clock
{
    private int _ticks;
    private TickEventHandler OnTickEvent;
    public void add_OnTickEvent(TickEventHandler h)
    { OnTickEvent += h; }
    public void remove_OnTickEvent(TickEventHandler h)
    { OnTickEvent -= h; }
    private void Tick(object sender, EventArgs e)
    {
        _ticks++;
        OnTickEvent?.Invoke(ticks, interval);
    }
}
\end{lstlisting}
\fat{Observer:}
\begin{lstlisting}
public class ClockObserver
{
    private string _name;
    public ClockObserver(string name)
    { _name = name; }
    public void OnTickEvent(int ticks, int i) 
    {
        Console.WriteLine("Observer {0} :
            Clock mit Interval {2}
            hat zum {1}. Mal getickt.", 
            _name, ticks, i);
    }
}
\end{lstlisting}
\fat{Test-Code:}
\begin{lstlisting}
public static void Test()
{
    Clock c1 = new Clock(1000);
    Clock c2 = new Clock(2000);

    ClockObserver t1 = new ClockObserver("O1");
    ClockObserver t2 = new ClockObserver("O2");

    //Observers anmelden
    c1.add_OnTickEvent(t1.OnTickEvent);
    c2.add_OnTickEvent(t2.OnTickEvent);

    // Warteschlaufe o.Ä

    //Observers abmelden
    c1.remove_OnTickEvent(t1.OnTickEvent);
    c2.remove_OnTickEvent(t2.OnTickEvent); 
}
\end{lstlisting}
\subsection{Events}
Reines Compiler-Feature / Syntactic Sugar. 
Das Delegate ist implizit private, damit es das Event nur von intern getriggert werden kann.
Subscribe / Unsubscribe Logik über +=, -=
\begin{lstlisting}
private TickEventHandler OnTickEvent;

// Compiler-Output
private TickEventHandler OnTickEvent;

public void add_OnTickEvent(TickEventHandler h)
{ OnTickEvent += h; }

public void remove_OnTickEvent(TickEventHandler h)
{ OnTickEvent -= h; }

//------------------------------------
//Delegate Syntax bei Events
public delegate void Name(object sender, EventArgs e);
//Rückgabetyp: void
//object sender: Absender übergibt bei Aufruf this mit
//EventArgs: Beliebige Sub-Klasse von EventArgs, Enthält Informationen zum Event
\end{lstlisting}
\subsubsection{Beispiel aus Callback vereinfacht}
\fat{Clock:}
\begin{lstlisting}
public delegate void TickEventHandler (int ticks, int interval);

public class Clock
{
    private int _ticks;
    public event TickEventHandler OnTickEvent;

    private void Tick(object sender, EventArgs e)
    {
        ticks++;
        OnTickEvent?.Invoke(ticks, interval);
    }
}
\end{lstlisting}
\fat{Oberver bleibt gleich wie bei Callback!}
\fat{Test-Code}
\begin{lstlisting}
public static void Test()
{
    Clock c1 = new(1000); Clock c2 = new(2000);
    ClockObserver t1 = new("O1");
    ClockObserver t2 = new("O2");
    
    //Observers anmelden
    c1.OnTickEvent += t1.OnTickEvent; 
    c2.OnTickEvent += t2.OnTickEvent;
    
    // Warteschlaufe o.Ä
    
    //Observers abmelden
    c1.OnTickEvent -= t1.OnTickEvent;
    c2.OnTickEvent -= t2.OnTickEvent;
}
\end{lstlisting}

\subsection{Lambdas}
\begin{itemize}
    \item In-place Angabe von Methodencode
    \item Keine Deklaration einer Methode nötig
\end{itemize}
\begin{lstlisting}
class Examples {
    void Test()
    {
        List<int> list = new();
        list.ForEach(i => Console.WriteLine(i));
        
        int sum = 0;
        list.ForEach(i => sum += 1; );
    }
}

//Expression Lambda
//(parameters) => expression
Func<int, bool> fe = i => i % 2 == 0;

//Statement Lambda
//(parameters) => { statements; } 
Func<int, bool> fs = i =>
{
    int rest = i%2;
    bool isRestZero = rest == 0;
    return isRestZero;
};
\end{lstlisting}
\subsubsection{Parameter}
\begin{itemize}
    \item Lambdas können 0-n Parameter haben
    \item Klammern um Parameter-Definition, ausser genau 1 Parameter und Tyü kann abgeleitet werden
    \item ref und out Parameter sind erlaubt
\end{itemize}
\begin{lstlisting}
bool IsEven(int x) { return x % 2 == 0; }
Func<int, bool> p01;
p01 = a => IsEven(a);
p01 = int a => IsEven(a); //Compilerfehler
p01 = (int a) => IsEven(a); //Ok, redundante Typdefinition
\end{lstlisting}
\includegraphics*[width=.5\columnwidth]{lambda}
\clearpage
\subsubsection{Closure / Outer Variables} 
\begin{itemize}
    \item Lambda Expression greift auf Variable der umgebenden Methode zu
    \item Variable wird in Hilfsobjekt ausgelagert (capturing):
    \item Hilfsobjekt lebt gleich lange wie Delegate-Objekt
\end{itemize}
\begin{lstlisting}
public delegate int Adder(); 

class Example
{
    static void Test()
    {
        Adder add = CreateAdder(); 
        Console.WriteLine(add()); 
        Console.WriteLine(add());
    }

    static Adder CreateAdder()
    {
        int x = 0;
        return () => ++x;
    }
}
//x wird in Hilfsobjekt ausgelagert
\end{lstlisting}
\end{multicols*}